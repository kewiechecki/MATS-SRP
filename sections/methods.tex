\section{Proposed Methods}
\subsection{My toy model of feature splitting}
I trained a MNIST classifier with a 3 neuron bottleneck.
I hypothesized that since the model had more labels than neurons in the bottleneck, features would be represented in superposition.
I refer to this as the ``outer model''.
Details are in Appendix \ref{app:outer}

%\subsection{Feature splitting}
I train three ``inner models'' to split the bottleneck activations into sparse features.
All of them have an embedding size of 27.
The PSAE is identical to the SAE except for containing a partitioner submodel.
Details are in Appendix \ref{app:inner}

%\end{multicols}

\begin{figure}
    \centering
     %\begin{subfigure}[b]{0.5\textwidth}
        \begin{tikzpicture}[
    node distance = 5mm and 5mm,
    punkt/.style = {rectangle, draw},
    pil/.style = {black, -stealth},
    font=\small
    ]

  %\node[punkt] (preprocessing) {preprocessing} ;
  \node[] (X) {$X^{m \times n}$} ;
  \node[punkt] (encoder) [below=of X] {$\encoder^{m \to d}$} ;
  \node[] (E) [below=of encoder] {$E^{d \times n}$} ;
  \node[punkt] (Etranspose) [below=of E] {\transpose} ;

  \node[punkt] (partitioner) [left=of encoder] {$\partitioner^{m \to k}$} ;
  \node[punkt] (softmax) [below=of partitioner] {\softmax} ;

  \node[] (K) [below=of softmax] {$K^{k \times n}$} ;
  %\node[punkt] (Ktranspose) [left=of K] {\transpose} ;
  %\node[punkt] (KK) [below=of Ktranspose] {\matmul} ;
  %\node[] (P) [below=of KK] {$P^{n \times n}$} ;
  \node[punkt] (wak) [below=of K] {\PWAK} ;
  \node[punkt] (GE) [below=of Etranspose] {\matmul} ;
  \node[] (Ehat) [below=of GE] {$\hat{E}^{d \times n}$} ;

  \node[punkt] (decoder) [left=of Ehat] {$\decoder^{d \to m}$} ;
  \node[] (Xhat) [left=of decoder] {$\hat{X}^{m \times n}$} ;

  \draw[pil] %(preprocessing) edge (X)
  (X) edge (encoder)
  (encoder) edge (E)
  (E) edge (Etranspose)
  (Etranspose) edge (GE)
  
  (X) edge (partitioner)
  (partitioner) edge (softmax)
  (softmax) edge (K)
  (K) edge (wak)
  %(Ktranspose) edge (KK)
  %(K) edge (KK)
  %(KK) edge (P)
  %(P) edge (wak)
  (wak) edge (GE)
  (GE) edge (Ehat)

  (Ehat) edge (decoder)
  (decoder) edge (Xhat);
\end{tikzpicture}


      %   \caption{}
      %   \label{fig:}
     %\end{subfigure}
     %\hfill
     
     \caption{
     $KE$ decomposition forward pass. 
     $E_{outer}$ is the activation matrix obtained from an outer model.
     For $n$ samples with $m$ activations, an inner encoder embeds each as a vector of length $d$.
     An inner classifier assigns probabilities for each sample being in each of $k$ clusters.
     Using the \PWAK transformation (see Appendix \ref{app:pwak}), 
     we obtain a denoising diffusion kernel based on the probability of sample pairs being in the same cluster.
     Matrix multiplication of the kernel with the inner embeddings gives denoised embeddings.
     The decoder unembeds the denoised embeddings back into activations of the outer model.
     The general method can be combined with any inner encoder, decoder, and classifier.
     See Appendix \ref{app:notation} for notation details.
     }
     \label{fig:deepwak}
\end{figure}

%\begin{multicols}{2}
    
%\subsection{Clustering}
I introduce two novel architectures that combine feature splitting with clustering. See Appendix \ref{} for details.
Both have identical clustering submodules with a maximum $k$ of 12.
They only differ is their encoders and (in the case of DeePWAK) decoders.

