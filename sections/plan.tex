%\begin{multicols}{2}

\section{Plan}

\subsection{Current Research Questions}
\begin{enumerate}
    \item Can a PSAE recover latent classifications from a toy model?
    \item Can PSAEs be used to aggregate SAE features?
    \item Are PSAE features more interpretable than SAE features?
    \item Can interpretability of a feature be defined in terms of effective information?
    \item Can we define a proxy metric for interpretability based on feature perturbation?
    \item Are cluster centroids from a PSAE representative of ``central cases'' of a classification?
    \item Can a PSAE detect formation of concepts during training?
    \item How does this compare to devinterp SOTA?
    \item Can PSAE clusters be used as a proxy for effective information gain during training?
\end{enumerate}

\subsection{Further Research Questions}
\begin{enumerate}
    \item How do linear combinations of PSAEs behave?
    \item How do sequential PSAEs differ from parallel PSAEs?
\end{enumerate}

\subsection{Big Questions}
\begin{enumerate}
    \item Is effective information a useful measure of model capability?
    \item Is it possible to have a model-independent measure of effective information of a data set?
    \item Are PSAEs a fully general clustering solution?
    \item Can a metaontology be expressed in terms of PSAEs?
\end{enumerate}

\begin{figure}
    \centering
     %\begin{subfigure}[b]{0.5\textwidth}
        \begin{tikzpicture}[
    node distance = 5mm and 5mm,
    punkt/.style = {rectangle, draw},
    pil/.style = {black, -stealth},
    font=\small
    ]

  %\node[punkt] (preprocessing) {preprocessing} ;
  \node[] (X) {$X^{m \times n}$} ;
  \node[punkt] (encoder) [below=of X] {$\encoder^{m \to d}$} ;
  \node[] (E) [below=of encoder] {$E^{d \times n}$} ;
  \node[punkt] (Etranspose) [below=of E] {\transpose} ;

  \node[punkt] (partitioner) [left=of encoder] {$\partitioner^{m \to k}$} ;
  \node[punkt] (softmax) [below=of partitioner] {\softmax} ;

  \node[] (K) [below=of softmax] {$K^{k \times n}$} ;
  %\node[punkt] (Ktranspose) [left=of K] {\transpose} ;
  %\node[punkt] (KK) [below=of Ktranspose] {\matmul} ;
  %\node[] (P) [below=of KK] {$P^{n \times n}$} ;
  \node[punkt] (wak) [below=of K] {\PWAK} ;
  \node[punkt] (GE) [below=of Etranspose] {\matmul} ;
  \node[] (Ehat) [below=of GE] {$\hat{E}^{d \times n}$} ;

  \node[punkt] (decoder) [left=of Ehat] {$\decoder^{d \to m}$} ;
  \node[] (Xhat) [left=of decoder] {$\hat{X}^{m \times n}$} ;

  \draw[pil] %(preprocessing) edge (X)
  (X) edge (encoder)
  (encoder) edge (E)
  (E) edge (Etranspose)
  (Etranspose) edge (GE)
  
  (X) edge (partitioner)
  (partitioner) edge (softmax)
  (softmax) edge (K)
  (K) edge (wak)
  %(Ktranspose) edge (KK)
  %(K) edge (KK)
  %(KK) edge (P)
  %(P) edge (wak)
  (wak) edge (GE)
  (GE) edge (Ehat)

  (Ehat) edge (decoder)
  (decoder) edge (Xhat);
\end{tikzpicture}


      %   \caption{}
      %   \label{fig:}
     %\end{subfigure}
     %\hfill
     
     \caption{
     $KE$ decomposition forward pass. 
     $E_{outer}$ is the activation matrix obtained from an outer model.
     For $n$ samples with $m$ activations, an inner encoder embeds each as a vector of length $d$.
     An inner classifier assigns probabilities for each sample being in each of $k$ clusters.
     Using the \PWAK transformation (see Appendix \ref{app:pwak}), 
     we obtain a denoising diffusion kernel based on the probability of sample pairs being in the same cluster.
     Matrix multiplication of the kernel with the inner embeddings gives denoised embeddings.
     The decoder unembeds the denoised embeddings back into activations of the outer model.
     The general method can be combined with any inner encoder, decoder, and classifier.
     See Appendix \ref{app:notation} for notation details.
     }
     \label{fig:deepwak}
\end{figure}

\subsection{Immediate goals}
\secsubhead{2-3 weeks}
\begin{enumerate}
    \item Train a toy model of feature splitting with checkpoints
    \item Implement an SAE and PSAE for the toy model
    \item Measure change in effective information during training
    \item Implement feature visualization
    \item Causal intervention on features
\end{enumerate}

\subsection{Initial paper}
\secsubhead{5 weeks}
\begin{enumerate}
    \item Review literature on interpretability metrics
    \item Determine how to quantify the interpretability of a feature
    \item Gain a better understanding of how causal emergence and SLT fit together
    \item Document git repo
    \item Separate writeup on theory \& how it fits into a broader alignment agenda
\end{enumerate}

\subsection{Medium term goals}
\secsubhead{4 months}
\begin{enumerate}
    \item Investigate bisemanticity: can we reconstruct a latent decision tree?
    \cite{aytekin2022neural}
    \item Adapt PSAE to GPT-2
    \item Determine if PSAE features contain more effective information than SAE features
    \item Compare effective information to techniques from 
    existing literature on developmental stages\cite{hoogland2024developmental}
\end{enumerate}

\subsection{6 month goals}
\begin{enumerate}
    \item Performance optimization
    \item Code distribution
    \item Toy model of illusory interpretability
    \item Develop metric for ``patchability'' of features
    \item Determine if ``patchability'' is a useful proxy for interpretability
\end{enumerate}

\subsection{1 year goals}
\begin{enumerate}
    \item Develop a toy model for learning an ontology
    \item Develop quality metrics for an ontology
    \item Test ontologizer architectures
\end{enumerate}

\subsection{Evaluation}

\paragraph{Cluster enrichment}
I perform a \hyperlink{https://en.wikipedia.org/wiki/Hypergeometric_distribution#Hypergeometric_test}{hypergeometric test}
for pairwise enrichment of all classifications in each model (Fig. \ref{fig:hyperMNIST}.
Clusters should correspond to the classifications of the outer model.

\paragraph{Effective information gain}
Clusters should become more degenerate over the course of (outer model) training.

\paragraph{Causal intervention}
How does perturbing a feature affect model behavior?
Is it easier to elicit desired behavior from PSAE features?

\paragraph{Adversarial activation patching}
Is it \textit{harder} to produce illusory interpretability by intervening on PSAE features?
It may be relevant that PSAE tends to produce fewer ``dead'' features (Fig. \ref{fig:E_MNIST}

\paragraph{Visual inspection}
Do cluster centroids correspond to recognizable features?

\subsection{Progress}
I attempt to determine whether an autoencoder combined with a self-supervised classifier produces more interpretable features than a na\"ive SAE.
I perform clustering with multiple architecture variants to assess stability of the latent classifications.
I try to assess whether a toy model trained on the same data but with different objectives learns the same abstractions.

\subsubsection{My toy model of feature splitting}
I trained a MNIST classifier with a 3 neuron bottleneck.
I hypothesized that since the model had more labels than neurons in the bottleneck, features would be represented in superposition.
I refer to this as the ``outer model''.
Details are in Appendix \ref{app:outer}.

%\subsection{Feature splitting}
I train three ``inner models'' to split the bottleneck activations into sparse features.
One is a canonical SAE. 
The others are variants on the architecture depicted in Fig. \ref{fig:deepwak}.
Details are in Appendix \ref{app:inner}.

See Appendix \ref{app:progress} for progress checklists.

%\end{multicols}
